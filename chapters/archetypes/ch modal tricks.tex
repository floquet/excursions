\chapter{Tricks}


  \begin{equation*}   %  =   =   =   =   =
   %\begin{split}
      y(x) = a_{0} + a_{1} x
   %\end{split}
   %\label{eq:}
  \end{equation*}

\section{A Single Line}  %    S    S    S    S    S    S    S    S    S
  \begin{equation}   %  =   =   =   =   =
   %\begin{split}
      y = a_{0} + a_{1} x
   %\end{split}
   \label{eq:myline}
  \end{equation}
Rewrite
  \begin{equation*}   %  =   =   =   =   =
   %\begin{split}
      -a_{1} x + y = a_{0}
   %\end{split}
   %\label{eq:}
  \end{equation*}

  \begin{equation*}   %  =   =   =   =   =
   %\begin{split}
      \mat{cc}{-a_{1} & 1} 
      \mat{c}{x\\y} =
      b
   %\end{split}
   %\label{eq:}
  \end{equation*}
  \begin{equation*}   %  =   =   =   =   =
   %\begin{split}
      x_{LS} = \lst{x\in\real{2}\colon \norms{\A{}x - b} \text{is minimized}}
   %\end{split}
   %\label{eq:}
  \end{equation*}
  \begin{equation*}   %  =   =   =   =   =
   %\begin{split}
      M\paren{x,y} = \paren{y - a_{0} - a_{1} x}^{2}
   %\end{split}
   %\label{eq:}
  \end{equation*}
Every point on the line is a solution point. Every point on the line produces the same error of 0. 
The least squares solution is the solution of minimum length.
  \begin{equation*}   %  =   =   =   =   =
   %\begin{split}
      y = -\frac{1}{a_{1}} x
   %\end{split}
   %\label{eq:}
  \end{equation*}
    \begin{equation*}   %  =   =   =   =   =
   %\begin{split}
      x = \frac{a_{0} a_{1}} {1 + a_{1}^{2}}
   %\end{split}
   %\label{eq:}
  \end{equation*}
  \begin{equation*}   %  =   =   =   =   =
   %\begin{split}
      p(\tau) = p_{0} + \tau \mat{c}{1\\a_{1}}
   %\end{split}
   %\label{eq:}
  \end{equation*}

  \begin{equation*}   %  =   =   =   =   =
   %\begin{split}
      \A{\dagger} x = \frac{a_{0}}{1+a_{1}^2} \mat{c}{-a_{1} \\ 1}
   %\end{split}
   %\label{eq:}
  \end{equation*}

Every point on the line in \eqref{eq:myline}

\section{Two Parallel Lines}  %    S    S    S    S    S    S    S    S    S
Where Do Parallel Lines Cross? The provocative question which opens this section has an obvious answer in Euclidean space. There is no such point; parallel lines never cross. Yet, if we input these lines as a linear system, we compute a least squares solution. What is the significance of the least squares solution?

Consider this tantalizing example. The two lines,
  \begin{equation*}   %  =   =   =   =   =
    \begin{split}
      y(x) &= \half - \half x, \\
      y(x) &= 1 - \half x,
    \end{split}
   %\label{eq:}
  \end{equation*}
are plotted in figure \ref{fig:tantalizing} along with the least squares solution is 
  \begin{equation*}   %  =   =   =   =   =
   %\begin{split}
      x_{LS} = \Ap b = \frac{1}{10}\mat{c}{ 3 \\ 6}.
   %\end{split}
   %\label{eq:}
  \end{equation*}
What is so special about this point?
\begin{figure}[htbp] %  figure placement: here, top, bottom, or page
   \centering
     \includegraphics[ width = 3in ]{\pathgraphics "tricks"/"parallel point"} 
   \caption{Parallel lines and the least squares solution.}
   \label{fig:tantalizing}
\end{figure}

\subsection{Intersecting Lines}
  \begin{equation*}   %  =   =   =   =   =
   \begin{split}
      y &= 1, \\
      y &= x .
   \end{split}
   %\label{eq:}
  \end{equation*}

  \begin{equation*}   %  =   =   =   =   =
   %\begin{split}
      \mat{rr}{0 & 1 \\ -1 & 1 }\mat{c}{x\\y} = \mat{c}{1\\0}
   %\end{split}
   %\label{eq:}
  \end{equation*}


\section{Removing Terms}  %    S    S    S    S    S    S    S    S    S
Here we develop a useful tool for removing isolated terms in the merit function.

Trial function for a line (polynomial expansion)
  \begin{equation*}   %  =   =   =   =   =
   %\begin{split}
      y_{k} = a_{0} + a_{1} x_{k}, \qquad k = 1, m
   %\end{split}
   %\label{eq:}
  \end{equation*}

In this instance, we can eliminate the intercept by looking at the differences between measurements. To wit,
  \begin{equation*}   %  =   =   =   =   =
   %\begin{split}
      y_{j} - y_{k} = a_{1} \paren{x_{j} - x_{k}}, \qquad \begin{cases}j=1, m-1\\ k= j+1, m\end{cases}
   %\end{split}
   \label{eq:difference jk}
  \end{equation*}
The common sense interpretation of this equation is that the difference in measurements tells us about the slope
Use the integer $p_{\nu}$, $\nu=\half n\paren{n-1}$ to keep track of the pairs of indices $(j,k)$. 
\begin{table}[htbp]
    \caption{default}
    \begin{center}
        \begin{tabular}{ll}
        	%
			$p_{1}$ & (1,2) \\
			$p_{2}$ & (1,3) \\
			\ $\vdots$ & \ $\vdots$ \\
			$p_{m-1}$ & $(1,m-1)$  \\
			$p_{m}$ & $(2,1)$  \\
			\ $\vdots$ & \ $\vdots$ \\
			$p_{\half n\paren{n-1}}$ & $(m-1,m)$  \\
        	%
        \end{tabular}
    \end{center}
    \label{default}
\end{table}
The variable $M  = \half n\paren{n-1}$ counts the number of distinct pairs.
  \begin{equation*}   %  =   =   =   =   =
   %\begin{split}
      Y_{p_{\mu}} = a_{1} Y_{p_{\mu}}, \qquad \mu = 1, M
   %\end{split}
   \label{eq:difference jk}
  \end{equation*}
The merit function is
  \begin{equation*}   %  =   =   =   =   =
   %\begin{split}
      M(a_{1}) = \sum_{\mu=1}^{M} \paren{Y_{p_{\mu}} - a_{1} X_{p_{\mu}}}^{2}
   %\end{split}
   %\label{eq:}
  \end{equation*}
Setting the derivative with respect to $a_{1}$ equal to zero generates the solution
  \begin{equation*}   %  =   =   =   =   =
   %\begin{split}
      a_{1} = \frac{\sum_{\mu=1}^{M} X_{p_{\mu}} Y_{p_{\mu}}} {\sum_{\mu=1}^{M} X_{p_{\mu}}^{2}}
   %\end{split}
   %\label{eq:}
  \end{equation*}

This is the same answer as \eqref{???},
`  \begin{equation*}   %  =   =   =   =   =
   %\begin{split}
      a_{1} = \frac{m \sum_{k=1}^{m}x_{k}y_{k} - \sum_{k=1}^{m}x_{k}\sum_{k=1}^{m}y_{k}} {m \sum_{k=1}^{m}x_{k}^{2} - \paren{\sum_{k=1}^{m}x_{k}}^{2}} .
   %\end{split}
   %\label{eq:}
  \end{equation*}
Numerator:
  \begin{equation*}   %  =   =   =   =   =
   %\begin{split}
      \sum_{j=1}^{m-1} \sum_{k=j+1}^{m} \paren{x_{j}y_{j} + x_{k}y_{k}} = (m-1) \sum_{i=1}^{m} x_{i}y_{i}
   %\end{split}
   \label{eq:rule 1}
  \end{equation*}
  \begin{equation*}   %  =   =   =   =   =
   %\begin{split}
   	  \sum_{i=1}^{m}x_{i} \sum_{i=1}^{m}y_{i} = \sum_{i=1}^{m}x_{i}y_{i} - 
      \sum_{j=1}^{m-1} \sum_{k=j+1}^{m} \paren{x_{j}y_{k} + x_{k}y_{j}} 
   %\end{split}
   \label{eq:rule 2}
  \end{equation*}


Denominator:

\section{Three Lines}  %    S    S    S    S    S    S    S    S    S
  \begin{equation*}   %  =   =   =   =   =
     \begin{split}
       %
       y_{1}(x) &= 1, \\
       y_{2}(x) &= 1 - x, \\
       y_{3}(x) &= m x.
       %
     \end{split}
   %\label{eq:}
  \end{equation*}

  \begin{equation*}   %  =   =   =   =   =
    \begin{split}
      &y = 1 \\
      x + &y = 1 \\
      -mx +&y = 0
    \end{split}
   %\label{eq:}
  \end{equation*}

  \begin{equation*}   %  =   =   =   =   =
   %\begin{split}
      \mat{rc}{0 & 1 \\ 1 & 1 \\ -m & 1 } 
      \mat{c}{x \\ y} = 
      \mat{c}{1\\1\\0}
   %\end{split}
   %\label{eq:}
  \end{equation*}

Merit function
  \begin{equation*}   %  =   =   =   =   =
   %\begin{split}
      M(x,y) = \normts{\A{}\mat{c}{x\\y}-b}
   %\end{split}
   %\label{eq:}
  \end{equation*}

  \begin{equation*}   %  =   =   =   =   =
   %\begin{split}
      \A{*} \cdot \A{} = 
      \mat{lc}{1+m^{2} & 1-m \\ 1-m & 3}
   %\end{split}
   %\label{eq:}
  \end{equation*}

  \begin{equation*}   %  =   =   =   =   =
    \begin{split}
      \tr {\A{}} &= 4 + m^{2} \\
      \det \paren{\A{}} &= 2 + 2m + 2m^{2}
    \end{split}
   %\label{eq:}
  \end{equation*}

  \begin{equation*}   %  =   =   =   =   =
   %\begin{split}
      p\paren{\lambda} = \lambda^{2} - \lambda \, \tr{\W{}} + \det\paren{\W{}}
   %\end{split}
   %\label{eq:}
  \end{equation*}
  \begin{equation*}   %  =   =   =   =   =
   %\begin{split}
      p\paren{\lambda} = 0
   %\end{split}
   %\label{eq:}
  \end{equation*}

  \begin{equation*}   %  =   =   =   =   =
   %\begin{split}
      \lambda_{\pm} = \frac{\tr{\W{}} \pm \sqrt{ \tr{\W{}}^{2} - 4 \det\paren{\W{}} } } {2}
   %\end{split}
   %\label{eq:}
  \end{equation*}


  \begin{equation*}   %  =   =   =   =   =
   %\begin{split}
      \sigma = 2^{-1/2} \sqrt{m^{2} + 4 \pm \factor}
   %\end{split}
   %\label{eq:}
  \end{equation*}

  \begin{equation*}   %  =   =   =   =   =
      \xi = \factor
   %\label{eq:}
  \end{equation*}


\begin{table}[htbp]
\caption{Least squares solution for three distinct lines as the parameter $m$ varies from 0 to $\infty$.}
    \begin{center}
        \begin{tabular}{ccc}
           %
           & Domain: Graphs & Domain: Merit Function \\\hline
           %
           \raisebox{1.5\height}{$m=0$} &
           \includegraphics[ width = 2in ]{\pathgraphics tricks/three_lines/"three lines m = 0"} &
           \includegraphics[ width = 2.1in ]{\pathgraphics tricks/three_lines/"merit function m = 0"} \\[10pt]
           %
           $m=1$ &
           \includegraphics[ width = 2in ]{\pathgraphics tricks/three_lines/"three lines m = 1"} &
           \includegraphics[ width = 2.1in ]{\pathgraphics tricks/three_lines/"merit function m = 1"} \\[10pt]
           %
           $m=2$ &
           \includegraphics[ width = 2in ]{\pathgraphics tricks/three_lines/"three lines m = 2"} &
           \includegraphics[ width = 2.1in ]{\pathgraphics tricks/three_lines/"merit function m = 2"} \\[10pt]
           %
           $m=5$ &
           \includegraphics[ width = 2in ]{\pathgraphics tricks/three_lines/"three lines m = 5"} &
           \includegraphics[ width = 2.1in ]{\pathgraphics tricks/three_lines/"merit function m = 5"} \\[10pt]
           %
           $m=\infty$ &
           \includegraphics[ width = 2in ]{\pathgraphics tricks/three_lines/"three lines m = inf"} &
           \includegraphics[ width = 2.1in ]{\pathgraphics tricks/three_lines/"merit function m = inf"} \\
           %
        \end{tabular}
    \end{center}
\label{tab:three lines}
\end{table}%


%merit function m = 0.pdf
%merit function m = 1.pdf
%merit function m = 2.pdf
%merit function m = 5.pdf
%merit function m = inf.pdfb
%three lines m = 0.pdf
%three lines m = 1.pdf
%three lines m = 2.pdf
%three lines m = 5.pdf
%three lines m = inf.pdf

\endinput