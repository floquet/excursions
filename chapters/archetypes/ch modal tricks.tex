\chapter{Tricks}

\section{Where Do Parallel Lines Cross?}  %    S    S    S    S    S    S    S    S    S
The provocative question which opens this section has an obvious answer in Euclidean space. There is no such point; parallel lines never cross. Yet, if we input these lines as a linear system, we compute a least squares solution. What is the significance of the least squares solution?

Consider this tantalizing example. The two lines,
  \begin{equation*}   %  =   =   =   =   =
    \begin{split}
      y(x) &= \half - \half x, \\
      y(x) &= 1 - \half x,
    \end{split}
   %\label{eq:}
  \end{equation*}
are plotted in figure \ref{fig:tantalizing} along with the least squares solution is 
  \begin{equation*}   %  =   =   =   =   =
   %\begin{split}
      x_{LS} = \Ap b = \frac{1}{10}\mat{c}{ 3 \\ 6}.
   %\end{split}
   %\label{eq:}
  \end{equation*}
What is so special about this point?
\begin{figure}[htbp] %  figure placement: here, top, bottom, or page
   \centering
     \includegraphics[ width = 3in ]{\pathgraphics "tricks"/"parallel point"} 
   \caption{Parallel lines and the least squares solution.}
   \label{fig:tantalizing}
\end{figure}

\subsection{Intersecting Lines}
  \begin{equation*}   %  =   =   =   =   =
   \begin{split}
      y &= 1, \\
      y &= x .
   \end{split}
   %\label{eq:}
  \end{equation*}

  \begin{equation*}   %  =   =   =   =   =
   %\begin{split}
      \mat{rr}{0 & 1 \\ -1 & 1 }\mat{c}{x\\y} = \mat{c}{1\\0}
   %\end{split}
   %\label{eq:}
  \end{equation*}


\section{Removing Terms}  %    S    S    S    S    S    S    S    S    S
Here we develop a useful tool for removing isolated terms in the merit function.

Trial function for a line (polynomial expansion)
  \begin{equation*}   %  =   =   =   =   =
   %\begin{split}
      y_{k} = a_{0} + a_{1} x_{k}, \qquad k = 1, m
   %\end{split}
   %\label{eq:}
  \end{equation*}

In this instance, we can eliminate the intercept by looking at the differences between measurements. To wit,
  \begin{equation*}   %  =   =   =   =   =
   %\begin{split}
      y_{j} - y_{k} = a_{1} \paren{x_{j} - x_{k}}, \qquad \begin{cases}j=1, m-1\\ k= j+1, m\end{cases}
   %\end{split}
   \label{eq:difference jk}
  \end{equation*}
The common sense interpretation of this equation is that the difference in measurements tells us about the slope
Use the integer $p_{\nu}$, $\nu=\half n\paren{n-1}$ to keep track of the pairs of indices $(j,k)$. 
\begin{table}[htbp]
    \caption{default}
    \begin{center}
        \begin{tabular}{ll}
        	%
			$p_{1}$ & (1,2) \\
			$p_{2}$ & (1,3) \\
			\ $\vdots$ & \ $\vdots$ \\
			$p_{m-1}$ & $(1,m-1)$  \\
			$p_{m}$ & $(2,1)$  \\
			\ $\vdots$ & \ $\vdots$ \\
			$p_{\half n\paren{n-1}}$ & $(m-1,m)$  \\
        	%
        \end{tabular}
    \end{center}
    \label{default}
\end{table}
The variable $M  = \half n\paren{n-1}$ counts the number of distinct pairs.
  \begin{equation*}   %  =   =   =   =   =
   %\begin{split}
      Y_{p_{\mu}} = a_{1} Y_{p_{\mu}}, \qquad \mu = 1, M
   %\end{split}
   \label{eq:difference jk}
  \end{equation*}
The merit function is
  \begin{equation*}   %  =   =   =   =   =
   %\begin{split}
      M(a_{1}) = \sum_{\mu=1}^{M} \paren{Y_{p_{\mu}} - a_{1} X_{p_{\mu}}}^{2}
   %\end{split}
   %\label{eq:}
  \end{equation*}
Setting the derivative with respect to $a_{1}$ equal to zero generates the solution
  \begin{equation*}   %  =   =   =   =   =
   %\begin{split}
      a_{1} = \frac{\sum_{\mu=1}^{M} X_{p_{\mu}} Y_{p_{\mu}}} {\sum_{\mu=1}^{M} X_{p_{\mu}}^{2}}
   %\end{split}
   %\label{eq:}
  \end{equation*}

This is the same answer as \eqref{???},
`  \begin{equation*}   %  =   =   =   =   =
   %\begin{split}
      a_{1} = \frac{m \sum_{k=1}^{m}x_{k}y_{k} - \sum_{k=1}^{m}x_{k}\sum_{k=1}^{m}y_{k}} {m \sum_{k=1}^{m}x_{k}^{2} - \paren{\sum_{k=1}^{m}x_{k}}^{2}} .
   %\end{split}
   %\label{eq:}
  \end{equation*}
Numerator:
  \begin{equation*}   %  =   =   =   =   =
   %\begin{split}
      \sum_{j=1}^{m-1} \sum_{k=j+1}^{m} \paren{x_{j}y_{j} + x_{k}y_{k}} = (m-1) \sum_{i=1}^{m} x_{i}y_{i}
   %\end{split}
   \label{eq:rule 1}
  \end{equation*}
  \begin{equation*}   %  =   =   =   =   =
   %\begin{split}
   	  \sum_{i=1}^{m}x_{i} \sum_{i=1}^{m}y_{i} = \sum_{i=1}^{m}x_{i}y_{i} - 
      \sum_{j=1}^{m-1} \sum_{k=j+1}^{m} \paren{x_{j}y_{k} + x_{k}y_{j}} 
   %\end{split}
   \label{eq:rule 2}
  \end{equation*}

Denominator:


\endinput