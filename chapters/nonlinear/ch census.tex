\chapter{Population Growth}

In this section we take a nonlinear model for population growth and separate the linear and nonlinear terms.

\section{Model}  %    S    S    S    S    S    S    S    S    S    S    S    S    S    S    S    S

% = =  e q u a t i o n
  \begin{equation}
    %\begin{split}
      y(\tau) = a_{1} + a_{2} \tau + a_{3} e^{d \tau}
    %\end{split}
    %\label{eqn:}
  \end{equation}
% = =

  \begin{equation*}   %  =   =   =   =   =
  %\begin{split}
    \A{}\paren{d + \gamma} \ne \A{}\paren{d} + \A{}\paren{\gamma}
    %\label{eq:}
  %\end{split}
  \end{equation*}

% https://tex.stackexchange.com/questions/278929/alignment-within-the-equation-environment
  \begin{equation*}   %  =   =   =   =   =
  \begin{array}{cccc}\arraycolsep = -0.5pt
  %
  %
  \A{}(d) & a & = & y \\
  %
    \mat{ccc}{
    1 & \tau_{1} & e^{d \tau_{1}} \\
    1 & \tau_{2} & e^{d \tau_{2}} \\
    1 & \tau_{3} & e^{d \tau_{3}} \\
    1 & \tau_{4} & e^{d \tau_{4}} \\
    1 & \tau_{5} & e^{d \tau_{5}} \\
    1 & \tau_{6} & e^{d \tau_{6}} \\
    1 & \tau_{7} & e^{d \tau_{7}} \\
    1 & \tau_{8} & e^{d \tau_{8}}
    } &
    \mat{c}{a_{0} \\ a_{1} \\ a_{3}} & = &
    \mat{c}{y_{0} \\ y_{1} \\ y_{3} \\ y_{4} \\ y_{5} \\ y_{6} \\ y_{7} \\ y_{8}}
    %\label{eq:}
  \end{array}
  \end{equation*}

% http://tex.stackexchange.com/questions/254650/aligning-multiple-things-under-a-limit
  \begin{equation} % = =  e q u a t i o n
    %\begin{split}
      \min_{\substack{a\in\real{3}\\d\in\real{}\phantom{^{3}}}} \normts{\A{}(d)\mat{c}{a_{1}\\a_{2}\\a_{3}}-y}
    %\end{split}
    %\label{eqn:}
  \end{equation}
\section{Problem Statement}  %    S    S    S    S    S    S    S    S    S    S    S    S

  \begin{equation}   %  =   =   =   =   =
  %\begin{split}
     r^{2} = \normts{\A{}(d)\mat{c}{a_{1}\\a_{2}\\a_{3}}-y}
    \label{eq:census:error}
  %\end{split}
  \end{equation}

  \begin{table}[t]  %  T A B L E
    \caption{Problem statement for population model with linear and exponential growth.}
    \begin{center}
      \begin{tabular}{lll}
        %
        \bf{trial function} & $y(\tau) = a_{0} + a_{1} \tau + a_{2} e^{d \tau}$ & $a\in\real{3}$ \\
        && $d\in\real{}$ \\
        \bf{merit function} & $M(a,d) = \sum\limits_{k=1}^{m}\paren{y_{k} - a_{1} + a_{2} \tau + a_{3} e^{d \tau_{k}}}^{2}$ \\
        \bf{\# measurements}& $m = 8$ \\
        \bf{\# parameters}  & $n = 4$ \\
        \bf{rank defect}    & $\rho = n$ & overdetermined \\
        \bf{input data}     & $\paren{\tau_{k}, y_{k}}$, $k=1\colon 8$ & table \ref{tab:census results}\\
        \bf{results}        & $a_{0}$ & constant\\
                            & $a_{1}$ & linear\\
                            & $a_{2}$ & exponential\\
                            & $d$      & power term\\
        \bf{residual error} & $r = \bl{\Ap b} - \Delta$ \\
        \bf{linear system}  & $\A{}(d)\mat{c}{a_{1}\\a_{2}\\a_{3}} = y$ \\
        %
      \end{tabular}
    \end{center}
  \label{tab:census problem statement}
  \end{table}%

\section{Data} %    S    S    S    S    S    S    S    S    S    S    S    S

\section{Example}  %    S    S    S    S    S    S    S    S    S    S    S    S    S    S    S    S
year $= 1900 + 10 (\tau - 1)$

    \begin{table}[t]
    	\caption{Data v. prediction.}
    	\begin{center}
    		\begin{tabular}{rrrrr}
    		%
		          &&&& \multicolumn{1}{c}{rel.}\\
    		 year & census & fit & \multicolumn{1}{c}{$r$} & error \\\hline
    		%
    		 1900 & 76.00  & 77.51  & 1.51  & 2.0\% \\
    		 1910 & 91.97  & 90.98  & $-0.99$ & $-1.1$\% \\
    		 1920 & 105.71 & 104.87 & $-0.84$ & $-0.8$\% \\
    		 1930 & 122.78 & 119.48 & $-3.29$ & $-2.7$\% \\
    		 1940 & 131.67 & 135.36 & 3.69  & 2.8\% \\
    		 1950 & 150.70 & 153.46 & 2.76  & 1.8\% \\
    		 1960 & 179.32 & 175.45 & $-3.87$ & $-2.2$\% \\
    		 1970 & 203.24 & 204.26 & 1.029 & 0.5\% \\
    		\end{tabular}
    	\end{center}
    	\label{tab:census results}
    \end{table}%


\begin{figure}[htbp] %  figure placement: here, top, bottom, or page
   \centering
   \begin{overpic}[ scale = \myscale ]
	   {\pathgraphics nonlinear/census/"census error wide"}
        %
    	\put(52,-3)  {$d$}
    	\put(-6,20) {\rotatebox{90}{$error, millions^{2}$}}
	    %
   \end{overpic}
      \caption{Residual error for \eqref{eq:census:error} with $a_{1}$, $a_{2}$, and $a_{3}$  at optimal values.}
   \label{fig:census:line:residual error:wide}
\end{figure}

\begin{figure}[htbp] %  figure placement: here, top, bottom, or page
   \centering
   \begin{overpic}[ scale = \myscale ]
	   {\pathgraphics nonlinear/census/"census error zoom"}
        %
    	\put(52,-3)  {$d$}
    	\put(-6,20) {\rotatebox{90}{$error, millions^{2}$}}
	    %
   \end{overpic}
      \caption{Residual error for $a_{1}$ and $a_{2}$ fixed at optimal values.}
   \label{fig:census:line:residual error:wide}
\end{figure}

\begin{figure}[htbp] %  figure placement: here, top, bottom, or page
   \centering
   \begin{overpic}[ scale = \myscale ]
	   {\pathgraphics nonlinear/census/"census data v soln"}
      %
    	%\put(53,-3) {$population, millions$}
    	\put(-6,19) {\rotatebox{90}{$population, millions$}}
	    %
   \end{overpic}
      \caption{Solution plotted against data.}
   \label{fig:census:data v soln}
\end{figure}

\begin{figure}[htbp] %  figure placement: here, top, bottom, or page
   \centering
   \begin{overpic}[ scale = \myscale ]
	   {\pathgraphics nonlinear/census/"census residual scatter"}
      %
    	%\put(52,-3)  {$d$}
    	\put(-6,20) {\rotatebox{90}{$error, millions^{2}$}}
	    %
   \end{overpic}
      \caption{Scatterplot of residual errors.}
   \label{fig:census:scatterplot}
\end{figure}

\begin{figure}[htbp] %  figure placement: here, top, bottom, or page
   \centering
   \begin{overpic}[ scale = \myscale ]
	   {\pathgraphics nonlinear/census/"census merit shaded"}
        %
    	\put(53,-3) {$a_{3}$}
    	\put(-4,49) {$d$}
	    %
   \end{overpic}
      \caption[The merit function showing least squares solution.]{The merit function with $a_{1}$ and $a_{2}$ fixed at best values showing least squares solution (center cross).}
   \label{fig:census:merit}
\end{figure}

\begin{figure}[htbp] %  figure placement: here, top, bottom, or page
   \centering
   \begin{overpic}[ scale = \myscale ]
	   {\pathgraphics nonlinear/census/"census merit lines"}
        %
    	\put(53,-3) {$a_{3}$}
    	\put(-4,49) {$d$}
	    %
   \end{overpic}
      \caption[The merit function showing least squares solution and the null cline.]{The merit function with $a_{1}$ and $a_{2}$ fixed at best values showing least squares solution (center cross) and the  (dashed line).}
   \label{fig:census:merit}
\end{figure}

    \begin{table}[t]
    	\caption{Results for census analysis}
    	\begin{center}
    		\begin{tabular}{ll}
    		  %
    		  \bf{fit parameters} & $c = \mat{r@{.}l}{0 & 010 \\ 0 & 0170 \\ 0 & 0096} \pm 
    		                             \mat{r@{.}l}{0 & 031 \\ 0 & 0014 \\ 0 & 0020}$ \\[18pt]
    		                      & $d = 0.056136\,\pm\,?.?$ \\[5pt]
    		  %
    		  $\rtr{*}$ & $55.12$\\
		      %
		      $\sum r_{k}$ & $-6.2 \times 10^{-14}$\\
		      %
		      $\sigma_{r}$ & $2.55$\\
    		  %
    		  $a$ & $\mat{r@{.}lr@{.}lr@{.}l}
    		    {0 & 5397 & \mg{-0} & \mg{0188} &  \mg{0} & \mg{0165} \\
    		    \mg{-0} & \mg{0188} &  0 & 0011 & \mg{-0} & \mg{0014} \\
    		     \mg{0} & \mg{0165} & \mg{-0} & \mg{0014} &  0 & 0022 }$\\[15pt]
    		  %
    		  \bf{plots} & data vs fit: figure \ref{fig:census:data v soln} \\
    		             & residuals: figure \ref{fig:census:scatterplot} \\
    		             & merit function in $\brnga{*}$: figure \ref{fig:census:merit} \\[5pt]
    		  %
    		\end{tabular}
    	\end{center}
    	\label{tab:results census}
    \end{table}%

\section{Polynomials} %    S    S    S    S    S    S    S    S    S    S    S    S
There is the model we choose and the model which nature chooses. Are they the same?

  \begin{table}[ht]  %  T A B L E
    \caption{Fitting the census data with low order polynomials.}
    \begin{center}
      \begin{tabular}{cc}
        %
        data and solution & residual errors\\
        %
        \includegraphics[ width = 2.25in ]{\pathgraphics nonlinear/census/"fit order = 0"} &
        \includegraphics[ width = 2.25in ]{\pathgraphics nonlinear/census/"residuals 0"} \\[5pt]
        %
        \includegraphics[ width = 2.25in ]{\pathgraphics nonlinear/census/"fit order = 1"} &
        \includegraphics[ width = 2.25in ]{\pathgraphics nonlinear/census/"residuals 1"} \\[5pt]
        %
        \includegraphics[ width = 2.25in ]{\pathgraphics nonlinear/census/"fit order = 2"} &
        \includegraphics[ width = 2.25in ]{\pathgraphics nonlinear/census/"residuals 2"} \\[5pt]
        %
        \includegraphics[ width = 2.25in ]{\pathgraphics nonlinear/census/"fit order = 3"} &
        \includegraphics[ width = 2.25in ]{\pathgraphics nonlinear/census/"residuals 3"} \\
        %
      \end{tabular}
    \end{center}
  %\label{tab:?}
  \end{table}%
  \begin{table}[ht]  %  T A B L E
    \caption{Fitting the census data with higher order polynomials.}
    \begin{center}
      \begin{tabular}{cc}
        %
        data and solution & residual errors\\
        %
        \includegraphics[ width = 2.25in ]{\pathgraphics nonlinear/census/"fit order = 4"} &
        \includegraphics[ width = 2.25in ]{\pathgraphics nonlinear/census/"residuals 4"} \\[5pt]
        %
        \includegraphics[ width = 2.25in ]{\pathgraphics nonlinear/census/"fit order = 5"} &
        \includegraphics[ width = 2.25in ]{\pathgraphics nonlinear/census/"residuals 5"} \\[5pt]
        %
        \includegraphics[ width = 2.25in ]{\pathgraphics nonlinear/census/"fit order = 6"} &
        \includegraphics[ width = 2.25in ]{\pathgraphics nonlinear/census/"residuals 6"} \\[5pt]
        %
        \includegraphics[ width = 2.25in ]{\pathgraphics nonlinear/census/"fit order = 7"} &
        \includegraphics[ width = 2.25in ]{\pathgraphics nonlinear/census/"residuals 7"} \\
        %
      \end{tabular}
    \end{center}
  %\label{tab:?}
  \end{table}%

%\subsection{Amplitudes}  %   SS   SS   SS   SS   SS   SS   SS   SS   SS   SS   SS   SS
%\begin{landscape}
%  \begin{table}[htbp]  %  T A B L E
%    \caption{Amplitudes by order of fit.}
%    \begin{center}
%      \begin{tabular}{ccccccccc}
%        %
%        order & \multicolumn{1}{c}{$a_{0}$} & \multicolumn{1}{c}{$a_{1}$} & \multicolumn{1}{c}{$a_{2}$} 
%              & \multicolumn{1}{c}{$a_{3}$} & \multicolumn{1}{c}{$a_{4}$} & \multicolumn{1}{c}{$a_{5}$} 
%              & \multicolumn{1}{c}{$a_{6}$} & \multicolumn{1}{c}{$a_{7}$} \\\hline
%        %
%%{\tiny{
%        $0$ & $133\pm41$ \\
%        %
%        $1$ & $-3260\pm190$ & $1.75\pm0.10$ \\
%        %
%        $2$ & $\paren{378\pm11}\times 10^{3}$ & $-41\pm11$ & $0.011\pm0.0029$ \\
%        %
%        $3$ & $\paren{-2.02\pm0.84}\times 10^{6}$ & $3100\pm1300$ & $-1.64\pm0.67$ & $\paren{2.8\pm1.2}\times 10^{-4}$ \\
%        %
%        $4$ & $\paren{-7.0\pm9.8}\times 10^{7}$ & $\paren{1.4\pm2.0}\times 10^{5}$ & $-110\pm160$ 
%            & $0.038\pm0.054$ & $\paren{-4.9\pm7.0}\times 10^{-6}$ \\
%        %
%        $5$ & $\paren{1.91\pm0.85}\times 10^{9}$ & $\paren{-4.9\pm2.2}\times 10^{7}$ & $\paren{5.1\pm2.3}\times 10^{4}$ 
%            & $-26\pm12$ & $\paren{6.8\pm3.0}\times 10^{-3}$  & $\paren{-7.1\pm3.1}\times 10^{-7}$\\
%        %
%        $7$ & $\paren{-1.5\pm1.1}\times 10^{12}$ & $\paren{4.7\pm3.2}\times 10^{9}$ & $\paren{-6.1\pm4.3}\times 10^{6}$ 
%            & $4200\pm2900$ & $-1.6\pm1.1$  & $\paren{3.4\pm2.4}\times 10^{-4}$  & $\paren{-2.9\pm2.0}\times 10^{-8}$\\
%        %
%%  }}
%      \end{tabular}
%    \end{center}
%  %\label{tab:?}
%  \end{table}
%\end{landscape}

\begin{figure}[htbp] %  figure placement: here, top, bottom, or page
   \centering
   \begin{overpic}[ scale = \myscale ]
	   {\pathgraphics nonlinear/census/"total error by order"}
        %
    \end{overpic}
      \caption{Total error $\rtr{*}$ by order of fit.}
   %\label{fig:census:line:residual error:wide}
\end{figure}
    
  \begin{table}[htbp]  %  T A B L E
    \caption{Projections, by order of fit, for population in 2010.}
    \begin{center}
      \begin{tabular}{cr}
        %
        order & population \\
        & (millions) \\\hline
        %
         0 & 133 \\
         1 & 264 \\
         2 & 320 \\
         3 & 420 \\
         4 & 300 \\
         5 & --928 \\
         6 & --4,540 \\
         7 & 22,183 \\
        %
      \end{tabular}
    \end{center}
  %\label{tab:?}
  \end{table}%    
\endinput