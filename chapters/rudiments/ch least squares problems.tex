\chapter{Least Squares Problems}

\section{Linear Systems}  %    S    S    S    S    S    S    S    S    S    S    S    S

  This story begins with the archetypal \index{matrix-vector equation} matrix--vector equation
  \begin{equation}   %  =   =   =   =   =
    \axeb .
    \label{eq:axeb}
  \end{equation}
The matrix $\A{}$ has $m$ rows, $n$ columns, and has rank $\rho$; the vector $b$ encodes $m$ measurements. The solution vector $x$ represents the $n$ free parameters in the model. In mathematical shorthand,
  \begin{equation}   %  =   =   =   =   =
    \aicmnr, \quad b \in \cmplxm, \quad x \in \cmplxn
  \label{eq:basics}
  \end{equation}
with $\cmplx{}$ representing the field of complex numbers. The system matrix $\A{}$ and the data vector $b$ are given, and the task is to find the solution vector $x$.

\subsection{$\reserr = 0$}  %   SS   SS   SS   SS   SS   SS   SS   SS   SS   SS   SS   SS
The letters in \eqref{eq:axeb} will change, but the operation remains the same: a matrix operates on an $n-$vector and returns an $m-$vector. We can think of the \index{matrix} matrix as a map from vectors of dimension $n$ to vectors of dimension $m$:
  \begin{equation*}   %  =   =   =   =   =
    \A{} \colon \cmplxn \mapsto \cmplxm .
  \end{equation*}

If the vector $b$ can be expressed a combination of the columns of the matrix $\A{}$ then there is a direct solution:
%  e q u a t i o n
\begin{equation*}
  %\begin{split}
    \axeb \quad \Longrightarrow \quad x_{1} a_{1} + \cdots + x_{n} a_{n} = b
%\label{eq:}
\end{equation*}
%  e q u a t i o n
and the residual error \index{residual error} vanishes:
  \begin{equation*}   %  =   =   =   =   =
      r = \axmb = \zero
  \label{eq:axmb}
  \end{equation*}
where the zero vector $\zero$ is a list of $m$ zeros. The total error, the norm of the $r$ vector, is 0:
  \begin{equation*}   %  =   =   =   =   =
      \normts{r} = \normts{\axmb} = 0.
  \end{equation*}

For the problem where the system matrix $\A{}$ is the identity matrix $\I{2}$:
  \begin{equation*}   %  =   =   =   =   =
      \idtwo \xtwo = \mat{c}{b_{1} \\ b_{2}},
 %\label{eq:}
  \end{equation*}
the solution is
  \begin{equation*}   %  =   =   =   =   =
      \xtwo = \mat{c}{b_{1} \\ b_{2}} ;
 %\label{eq:}
  \end{equation*}
there is no residual error 
  \begin{equation*}   %  =   =   =   =   =
      r = \A{} x - b = \zerotwo .
 %\label{eq:}
  \end{equation*}

\subsection{$\reserr > 0$}  %   SS   SS   SS   SS   SS   SS   SS   SS   SS   SS   SS   SS
But what happens when the vector $b$ is \emph{not} in the column space of the matrix $\A{}$? The solution criteria must relax. Instead of seeking zero residual error, seek the least residual error
  \begin{equation*}   %  =   =   =   =   =
      \norm{r} = \reserr.
  \end{equation*}
Instead of a perfect solution, ask for the best solution. One such class of solutions are least squares solutions\index{leasts squares!solutions}.

\section{Least Squares Solutions}  %    S    S    S    S    S    S    S    S    S    S    S    S    S    S
In computation of approximations, the goal is to minimize the residual error and this work explores the minimal solutions under the $2-$norm, the familiar norm of Pythagoras:
  \begin{equation*}   %  =   =   =   =   =
      \normt{ \mat{c}{x_{1} \\ x_{2}} } = \sqrt{x_{1}^{2} + x_{2}^{2}.}
 %\label{eq:}
  \end{equation*}

Let's construct a sample problem with $\reserr > 0$ by modifying the previous example:
  \begin{equation}   %  =   =   =   =   =
      \mat{cc}{1 & 0 \\ 0 & 0} \xtwo = \mat{c}{b_{1} \\ b_{2}} .
      \label{eq:simple case}
  \end{equation}
When $b_{2} \ne 0$ there is no exact solution, a solution with $\reserr = 0$. Consider the solutions given by
  \begin{equation}   %  =   =   =   =   =
      x_{*} = \mat{c}{ x_{1} \\ 0 } ;
      \label{eq:xstar}
  \end{equation}
the error is $\A{} x_{*} - b = -\mat{c}{0 \\ b_{2}}$ which has a norm 
  \begin{equation}   %  =   =   =   =   =
      \normt{r} = \normt{\A{}x_{*} - b} = \abs{b_{2}},
  \label{eq:simple case solution}
  \end{equation}
plotted in figure \ref{fig:v graph}. This is the least possible error for the problem and \eqref{eq:xstar} is the best solution. The transition from an exact solution to an inexact solution is continuous.
\begin{figure}[htbp] %  figure placement: here, top, bottom, or page
   \centering
   \begin{overpic}[ scale = \myscale ]
	   {\pathgraphics "least squares"/"abs error"}
		% 
    	\put(51,-2){$b_2$}
    	\put(-8,30){$\normt{r}$}
	    %
   \end{overpic}
   \caption{The residual error $\normt{r}$ given in \eqref{eq:simple case solution}.}
   \label{fig:v graph}
\end{figure}

Think of the solutions to the linear system
  \begin{equation*}   %  =   =   =   =   =
    \axeb
  \end{equation*}
as being described by the inequality
  \begin{equation*}   %  =   =   =   =   =
    \normt{\axmb} \ge 0.
  \end{equation*}
The equality is attained when the data vector lies within the column space of $\A{}$.

Least squares solutions are classified by the interpretation of the output. In the first case, \emph{zonal approximation}, the output represents data at a physical zone, a point or a region. In the second case, \emph{modal approximation}, the output represents an amplitude, a contribution for a mode. Basic examples follow.

\subsection{Zonal Approximation}  %   SS   SS   SS   SS   SS   SS   SS   SS   SS   SS   SS   SS
Consider the vector field $F$ described by the gradient of a scalar field $\phi$.
  \begin{equation*}   %  =   =   =   =   =
    F = \nabla \phi
  \end{equation*}
The forward problem involves taking a given potential $\phi$ and computing the forces $F$. We are instead interested in the inverse problem: measure the forces $F$ and compute the potential $\phi$.

\subsubsection{\label{ssec:zonal solution}Zonal Problem}  %  SSS  SSS  SSS  SSS  SSS  SSS  SSS  SSS  SSS  SSS  SSS  SSS
In practice one measures the vector field and solves the inverse problem by computing values for the potential at the zone boundaries. The input and outputs are represented in \ref{fig:sticks}. The physical field is $\phi(x)$, $0\le x \le 2$, the approximation is $\varphi_{x_{k}}$, $k=0,1,2$. For a cleaner presentation let $\varphi_{x_{k}} \rightarrow \varphi_{k}$. The first measurement $x_{1}$ represents the potential change between $\phi(0)$ and $\phi(1)$; the second measurement $x_{2}$ the change between $\phi(1)$ and $\phi(2)$.
  \begin{equation*}   %  =   =   =   =   =
    \begin{split}
      \varphi_{1} - \varphi_{0} &\approx \delta_{1} \\
      \varphi_{2} - \varphi_{1} &\approx \delta_{2}
    %\label{eq:}
    \end{split}
  \end{equation*}

\begin{figure}[htbp] %  figure placement: here, top, bottom, or page
   \centering
   \begin{overpic}[ scale = \myscale ]
	   {\pathgraphics "least squares"/"zonal"}
   \end{overpic}
   \caption[Scalar function $\phi$ and approximations.]{Scalar function $\phi(x)$ (curve) and approximation $\varphi_{k}$ (sticks).}
   \label{fig:sticks}
\end{figure}

The system matrix 
  \begin{equation*}   %  =   =   =   =   =
      \A{} =     \mat{rrr}{ 
      -1 & 1 & 0 \\
       0 & -1 & 1 } \in \real{2 \times 3}_{2}.
 %\label{eq:}
  \end{equation*}
There are $m = 2$ measurements, $n = 3$ measurement locations, and the matrix rank is $\rho = 2$. Because the rank is less than the number of columns, $\rho < n$, this problem is \emph{underdetermined}.\index{underdetermined system}

The linear system is
  \begin{equation}   %  =   =   =   =   =
    \mat{rrr}{ 
      -1 & 1 & 0 \\
       0 & -1 & 1 }
    \mat{c}{ \varphi_{0} \\ \varphi_{1} \\ \varphi_{2} }
    =
    \mat{c}{ \delta_{1} \\ \delta_{2} } .
    \label{eq:zonalls}
  \end{equation}

\subsubsection{Zonal Solution}  %  SSS  SSS  SSS  SSS  SSS  SSS  SSS  SSS  SSS  SSS  SSS  SSS

The solutions for the linear system in \eqref{eq:zonalls} which minimize $\normt{\axmb}$ are
  \begin{equation*}   %  =   =   =   =   =
    \mat{c}{ \varphi_{0} \\ \varphi_{1} \\ \varphi_{2} } =
    \bl{\mat{rr}{ -2 & -1 \\ 1 & -1 \\  1 & 2 }
    \mat{c}{ \delta_{1} \\ \delta_{2} }} + \gamma
    \rd{\mat{c}{ 1 \\ 1 \\ 1 }}, \qquad \gamma \in \cmplx{} .
  \end{equation*}
The color blue represents range space vectors, red null space vectors. In this way, the fundamental spaces spring to life.

There is a continuum of solutions due to the fact that
  \begin{equation*}   %  =   =   =   =   =
      \A{} \paren{x + \gamma \rd{\mat{c}{ 1 \\ 1 \\ 1 } } } = \A{}x + \A{} \paren{\gamma \rd{\mat{c}{ 1 \\ 1 \\ 1 } } } = \A{}x + \gamma \A{} \rd{\mat{c}{ 1 \\ 1 \\ 1 } } = \Ax.
  \end{equation*}
An quick demonstration confirms the red vector is in the null space: 
  \begin{equation*}   %  =   =   =   =   =
    \A{} \rd{\mat{c}{ 1 \\ 1 \\ 1 }} 
      = \mat{rrr}{ -1 & 1 & 0 \\ 0 & -1 & 1 } \rd{\mat{c}{ 1 \\ 1 \\ 1 }} 
      = \rd{ \mat{c}{ 0 \\ 0 } }.
  \end{equation*}

\subsection{\label{ssec:modal approx}Modal Approximation}  %   SS   SS   SS   SS   SS   SS   SS   SS   SS   SS   SS   SS
\subsubsection{\label{ssec:modal problem}Modal Problem}  %  SSS  SSS  SSS  SSS  SSS  SSS  SSS  SSS  SSS  SSS  SSS  SSS
In the modal approximation, the user first selects a set of basis functions to describe measurements. Popular basis functions include orthogonal polynomials, trigonometric functions, or monomials. For example, a linear regression implies a basis set of two elements: a constant function, and a linear function. This leads to the familiar equation for a straight line:
  \begin{equation*}   %  =   =   =   =   =
    y(x) = a_{0} + a_{1} x
  \end{equation*}
The $n=2$ parameters represent the intercept $\paren{a_{0}}$, and the slope $\paren{a_{1}}$; each of the $m$ measurements represents a straight line:
  \begin{equation*}   %  =   =   =   =   =
   \begin{split}
     a_{0} + a_{1} x_{1} &= y_{1} \\
       & \ \, \vdots \\
     a_{0} + a_{1} x_{m} &= y_{m} .
    %\label{eq:}
   \end{split}
  \end{equation*}
The goal is to simultaneously solve the set of equations.

\subsubsection{\label{ssec:modal solution}Modal Solution}  %  SSS  SSS  SSS  SSS  SSS  SSS  SSS  SSS  SSS  SSS  SSS  SSS
The first step is to compose the system
  \begin{equation}   %  =   =   =   =   =
   \begin{array}{cccc}
     \A{} & \alpha & = & y \\
     \mat{cc}{ 1 & x_{1} \\ \vdots & \vdots \\ 1 & x_{m} } &
     \mat{c}{ \alpha_{0} \\ \alpha_{1} } &
       = &
     \mat{c}{ y_{1} \\ \vdots \\ y_{m} } ,
    \label{eq:modalls}
   \end{array}
  \end{equation}
which can be expressed using the column vectors
  \begin{equation*}   %  =   =   =   =   =
   \begin{split}
     {\bf{1}} = \mat{cc}{ 1 \\ \vdots \\ 1 }, \quad
     x = \mat{cc}{ x_{1} \\ \vdots \\ x_{m} }, \quad
     y = \mat{cc}{ y_{1} \\ \vdots \\ y_{m} } .
    %\label{eq:}
   \end{split}
  \end{equation*}
The columns of the system matrix $\A{} = \mat{c|c}{\bf{1} & x}$. The solution parameters can be expressed in terms of the column vectors:
  % = =  e q u a t i o n
  \begin{equation}
    \mat{c}{\alpha_{0} \\ \alpha_{1}} =
    \paren{ \paren{\oto}\paren{\xtx} - \paren{\otx}^{2}}^{-1}
     \bl{\mat{rr}{\xtx & -\otx \\
            -\otx &  \oto }
    \mat{c}{ \mathbf{1}^\mathrm{T}y \\  x^\mathrm{T}y}} .
    \label{eq:column vectors}
  \end{equation}
  % = =

\subsection{Errors}  %   SS   SS   SS   SS   SS   SS   SS   SS   SS   SS   SS   SS
When measurements are not exact, solutions are not exact. A great beauty of the method of least squares in that the quality of the solution can be quantified. An ability to discern answers like 3.0 $\pm$ 1.0 from 3.000 $\pm$ 0.0010 is invaluable. The machinery needed to compute uncertainties will be developed in following chapters.

\section{\label{sec:lsp}General Least Squares Problem and Solution}  %    S    S    S    S    S    S    S    S    S    S    S    S    S    S
Emboldened by solutions to two basic problems, we turn attention towards formalities. Starting with a the linear system $\axeb$ where the matrix $\aicmn$, the data vector $b\in\cmplxm$, the least squares solution\index{least squares!solution!definition} $\xls$ is defined as the set
  \begin{equation}   %  =   =   =   =   =
  \boxed{
    \xlsdef .
    \label{eq:xlsdef}
	}
  \end{equation}
The least squares solution may be a point or it may be a hyperplane. The general solution is a combination of a particular solution (range space component in blue) and a homogenous solution (null space component in red):
  \begin{equation}   %  =   =   =   =   =
  \boxed{
   \begin{array}{cccccl}
     \xls 
       & = & \bsolnlsa{b}{y}, & \qquad y \in \cmplx{n} \\
       & = & \xbotha, \\
       & = & \bl{x_{part}} & + & \rd{x_{homog}} .
    \label{eq:general soln}
   \end{array}
  }
  \end{equation}
where the matrix $\Ap$ is the pseudoinverse and $y$ is an arbitrary vector.


\endinput