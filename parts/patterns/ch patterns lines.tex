\chapter{Lines}

\section{Face-centered cubic lattice} %    S    S    S    S    S    S    S    S    S    S    S    S

\begin{figure}[htbp] %  figure placement: here, top, bottom, or page
   \centering
   \includegraphics[ width = 3in ]{\pathgraphics "patterns"/"lines"/"fcc lattice with cell"} 
   \caption{A slice of a face-centered cubic lattice showing a single crystal.}
   \label{fig:pattern fcc}
\end{figure}

\begin{figure}[htbp] %  figure placement: here, top, bottom, or page
   \centering
   \includegraphics[ width = 3in ]{\pathgraphics "patterns"/"lines"/"potentials"} 
   \caption{Simulation output showing atomic shades shaded by potential energy.}
   \label{fig:pattern potentials}
\end{figure}

\section{Model}  %    S    S    S    S    S    S    S    S    S    S    S    S

\begin{equation}
  %\begin{split}
    y_{(\mu)}(x) = \mu \alpha_{*} + \alpha_{0} + \alpha_{1} x, \qquad \mu = 0,1,2,\dots,M-1 .
    \label{eq:trial coupled}
  %\end{split}
\end{equation}

  \begin{equation}
    \begin{split}
    %
      \begin{cases}
        0\cdot\alpha_{*} + \alpha_{0} + \alpha_{1} x_{1_{1}} = y_{1_{1}} \\
        \qquad \qquad \qquad \qquad \quad \vdots \\
        0\cdot\alpha_{*} + \alpha_{0} + \alpha_{1} x_{m_{1}} = y_{m_{1}} \\
      \end{cases}
      row\ 1 \\
    %
      \begin{cases}
        1\cdot\alpha_{*} + \alpha_{0} + \alpha_{1} x_{1_{2}} = y_{1_{2}} \\
        \qquad \qquad \qquad \qquad \quad \vdots \\
        1\cdot\alpha_{*} + \alpha_{0} + \alpha_{1} x_{m_{2}} = y_{m_{2}} \\
      \end{cases}
      row\ 2 \\
    %
      \begin{cases}
        2\cdot\alpha_{*} + \alpha_{0} + \alpha_{1} x_{1_{3}} = y_{1_{3}} \\
        \qquad \qquad \qquad \qquad \quad \vdots \\
        2\cdot\alpha_{*} + \alpha_{0} + \alpha_{1} x_{m_{3}} = y_{m_{3}} \\
      \end{cases}
      row\ 3 \\
    %
    \end{split}
  \label{eq:system}
  \end{equation}

\begin{figure}[htbp] %  figure placement: here, top, bottom, or page
   \centering
   \includegraphics[ width = 3in ]{\pathgraphics "patterns"/"lines"/"all with inset"} 
   \caption{Full data set showing inset.}
   \label{fig:simulation inset}
\end{figure}

\begin{figure}[htbp] %  figure placement: here, top, bottom, or page
   \centering
   \includegraphics[ width = 4in ]{\pathgraphics "patterns"/"lines"/"parameters"} 
   \caption{Sample data set showing fit parameters.}
   \label{fig:parameters}
\end{figure}
\clearpage

  \begin{table}[t]  %  T A B L E
    \caption{Data sets and basic results}
    \begin{center}
      \begin{tabular}{lc c r@{.}l c r@{.}l}
        %
        \multirow{4}{*}{
        \includegraphics[ width = 2.7in ]{\pathgraphics "patterns"/"lines"/"data set rows"}}
        & $\alpha_{*}$ & = & $0$ & $9899$ & $\pm$ & $0$ & $0032$ \\
        & $\alpha_{0}$ & = & $3$ & $438$  & $\pm$ & $0$ & $013$ \\
        & $\alpha_{1}$ & = & $0$ & $3376$ & $\pm$ & $0$ & $0017$ \\[5pt]
        & $\sqrt{\inner{r^{2}}}$ & = & $0$ & $052$ \\[80pt]
        %
        \multirow{4}{*}{
        \includegraphics[ width = 2.7in ]{\pathgraphics "patterns"/"lines"/"data set cols"}}
        & $\alpha_{*}$ & = & $4$  & $974$ & $\pm$ & $0$ & $052$ \\
        & $\alpha_{0}$ & = & $-2$ & $075$ & $\pm$ & $0$ & $093$ \\
        & $\alpha_{1}$ & = & $5$  & $168$ & $\pm$ & $0$ & $052$ \\[5pt]
        & $\sqrt{\inner{r^{2}}}$ & = & $0$ & $18$ \\[80pt]
        %
        \multirow{4}{*}{
        \includegraphics[ width = 2.7in ]{\pathgraphics "patterns"/"lines"/"data set spokes"}}
        & $\alpha_{*}$ & = & $1$  & $2322$ & $\pm$ & $0$ & $0039$ \\
        & $\alpha_{0}$ & = & $-7$ & $505$  & $\pm$ & $0$ & $043$ \\
        & $\alpha_{1}$ & = & $-0$ & $8576$ & $\pm$ & $0$ & $0038$ \\[5pt]
        & $\sqrt{\inner{r^{2}}}$ & = & $0$ & $054$
        %
      \end{tabular}
    \end{center}
  \label{tab:results grains}
  \end{table}%

\clearpage

  \begin{equation}
    \begin{array}{cccc}
      \A{} & \alpha & = & y \\
      \mat{ccc}{ 
      0 & 1 & x_{1_{1}} \\ \vdots & \vdots & \vdots \\ 
      0 & 1 &  x_{\mu_{1}} \\\arrayrulecolor{medgray}\hline
      1 & 1 & x_{1_{2}} \\ \vdots & \vdots & \vdots \\ 
      1 & 1 &  x_{\mu_{2}} \\\hline
      2 & 1 & x_{1_{3}} \\ \vdots & \vdots & \vdots \\ 
      2 & 1 &  x_{\mu_{3}} \\\hline
      \vdots & \vdots & \vdots \\\hline
      M - 1 & 1 & x_{1_{M}} \\ \vdots & \vdots & \vdots \\ 
      M - 1 & 1 &  x_{m_{M}} \\
      } &
      \mat{c}{ \alpha_{*} \\ \alpha_{0} \\ \alpha_{1} } & = &
      \mat{c}{ 
      y_{1_{1}} \\ \vdots \\ y_{\mu_{1}} \\\arrayrulecolor{medgray}\hline 
      y_{1_{2}} \\ \vdots \\ y_{\mu_{2}} \\\hline 
      y_{1_{3}} \\ \vdots \\ y_{\mu_{3}} \\\hline 
      \vdots\\\hline
      y_{1_{M}} \\ \vdots \\ y_{\mu_{M}}
      }
    \end{array}
    \label{eq:lines A}
  \end{equation}

\section{Solution}  %    S    S    S    S    S    S    S    S    S    S    S    S
Once again the normal equations offer the easy path to solution as in \eqref{eq:normal solution}. The first step is to compute the inverse of the product matrix. Recall that the dot product is a commutative operator; therefore only six of the nine matrix entries are unique: 
% = =  e q u a t i o n
\begin{equation*}
  %\begin{split}
    \wx{T} = \mat{ccc}{
    \mathrm{J} \cdot \mathrm{J} & \mathrm{J} \cdot \mathbf{1} & \mathrm{J} \cdot x \\
    \mathbf{1} \cdot \mathrm{J} & \mathbf{1} \cdot \mathbf{1} & \mathbf{1} \cdot x \\
    x\cdot \mathrm{J} & x\cdot\mathbf{1} & x\cdot x \\
    }= \mat{ccc}
    {
      a & b & c \\
      \mg{b} & d & e \\
      \mg{c} & \mg{e} & f
    } .
    \label{eq:mnormal}
  %\end{split}
\end{equation*}
% = =
For clarity, the unique elements are specified:
\begin{align*}
  a   & = \mathrm{J} \cdot \mathrm{J} & b & = \mathrm{J} \cdot \mathbf{1} & c& = \mathrm{J} \cdot x \nonumber\\
  && d& = \mathbf{1} \cdot \mathrm{J} & e & = \mathbf{1} \cdot x \\ %\label{eq:abc}\\
  &&&& f& = x \cdot x\nonumber
\end{align*}
In advance of the computing the inverse, first compute the determinant
% = =  e q u a t i o n
\begin{equation*}
  %\begin{split}
    \det \paren{\wx{T}} = \Delta = 2 b c e + a d f - a e^{2} - c^{2} d  - f b^{2} .
    \label{eq:det}
  %\end{split}
\end{equation*}
% = =
Using \eqref{eq:minverse} the inverse is
% = =  e q u a t i o n
\begin{equation*}
  %\begin{split}
    \paren{\wx{T}}^{-1} = \Delta^{-1}
    \mat{ccc}{
       d f - e^{2} & c e - b f & b e - c d \\
       \cdot & a f - c^{2} & b c - a e \\
       \cdot & \cdot & a d - b^{2} \\
    } .
    %\label{eqn:}
  %\end{split}
\end{equation*}
% = =
The right-hand side in \eqref{eq:mnormal} is
% = =  e q u a t i o n
\begin{equation*}
    \A{T}y = \beta = 
      \mat{c}{ \beta_{1} \\ \beta_{2} \\ \beta_{3} } = 
      \mat{c}{ \mathrm{J}\cdot y \\ \mathbf{1}\cdot y \\ x\cdot y } .
    \label{eq:rhs}
\end{equation*}
% = =

The least squares solution is provided as
% = =  e q u a t i o n
\begin{equation*}
  %\begin{split}
    \mat{c}{ \alpha_{0} \\ \alpha_{*} \\ \alpha_{1} } = \paren{\wx{T}}^{-1} \A{T}y 
    \label{eq:soln3}
  %\end{split}
\end{equation*}
% = =
which distills down to
% = =  e q u a t i o n
\begin{equation*}
  %\begin{split}
    \alpha = \Delta^{-1} 
    \left[
    \begin{array}{r@{ + }c@{ + }l}
      \beta_{1} \paren{d f -  e^{2}} \phantom{i} &  \phantom{i} \beta_{2} \paren{c e - b f} \phantom{i}   & \phantom{i} \beta_{3} \paren{b e - c d} \\
      \beta_{1} \paren{c e - b f} \phantom{i}    &  \phantom{i} \beta_{2} \paren{a f - c^{2}} \phantom{i} & \phantom{i} \beta_{3} \paren{b c - a e} \\
      \beta_{1} \paren{b e - c d} \phantom{i}    &  \phantom{i} \beta_{2} \paren{b c - a  e} \phantom{i}  & \phantom{i} \beta_{3} \paren{a d - b^{2}}
    \end{array}
    \right] .
    \label{eq:error3}
  %\end{split}
\end{equation*}
% = =
The errors associate with the fit parameters are
  % = =  e q u a t i o n
  \begin{equation*}
    \mat{c}{\sigma_{*} \\ \sigma_{0} \\ \sigma_{1}} = \sqrt{\frac{r^{\mathrm{T}}}{\paren{m-n}\Delta}}
    \sqrt{\mat{c}{ d f - e^{2} \\ a f - c^{2} \\ a d - b^{2} }} .
    \label{eq:errors2}
  \end{equation*}
  % = =
The solutions are expressed in terms of dot products readily available in Fortran.


\section{Problem Statement}  %    S    S    S    S    S    S    S    S    S    S    S    S

  \begin{table}[t]  %  T A B L E
    \caption{Problem statement for grain identification by rows (coupled linear regression).}
    \begin{center}
      \begin{tabular}{lll}
        %
        \bf{trial function} & $y_{(\mu)}(x) = \mu \alpha_{*} + \alpha_{0} + \alpha_{1} x$ \\
        \bf{merit function} & $M(p) = \sum_{k=1}^{n}\paren{y_{k} - \mu \alpha_{*} + \alpha_{0} + \alpha_{1} x_{k}}^{2}$ \\
        \bf{number of zones}& $m = 5$ \\
        \bf{number of overlaps}& $n = 4$ \\
        \bf{rank defect}    & $m - n  = 1$ \\
        \bf{measurements}   & $\lambda = \lst{11, 13, 13, 13, 12}$ \\
        \bf{measurements}   & $\paren{x_{k}, y_{k}}$, $k=1\colon 1024$ \\
        \bf{system matrix}  & $\A{} = \mat{cc}{\mathbf{1} & x}$ \\
        \bf{data vector}    & $y$ \\
        \bf{linear system}  & \eqref{eq:lines A} \\
        
        \bf{results}        & $\alpha_{*}$ & gap\\
                            & $\alpha_{0}$ & $y-$axis intercept\\
                            & $\alpha_{1}$ & slope\\
        \bf{residual error} & $r = \A{} \alpha - y$ \\
        %
      \end{tabular}
    \end{center}
  \label{tab:find lines problem statement}
  \end{table}%

\section{Data} %    S    S    S    S    S    S    S    S    S    S    S    S
  
%        T A B L E
{\tiny{
\begin{table}[htbp]
\caption{Point membership in data sets shown in figure \ref{tab:results grains}.}
  \begin{center}
    \begin{tabular}{cccccccccccccccc}
      %
      % user: rditldmt, CPU: dan-topas-pro-2, MM v. 10.0 for Mac OS X x86, date: \mathrm{J}un 3, 2015, time: 11:04:41, nb: /Users/rditldmt/Dropbox/ nb/drc/molecular dynamics/orientation/dots and numbers 04.nb
      %
      set & row & 1 & 2 & 3 & 4 & 5 & 6 & 7 & 8 & 9 & 10 & 11 & 12 & 13 & 14 \\\hline
      %
      1 &  1 & 519 & 520 & 522 & 555 & 556 & 589 & 590 & 591 & 624 & 625 \\
      %
      1 &  2 & 551 & 552 & 553 & 554 & 587 & 588 & 621 & 622 & 623 & 656 & 657 & 658 \\
      %
      1 &  3 & 582 & 583 & 584 & 585 & 586 & 619 & 620 & 653 & 654 & 655 & 688 & 689 & 690 \\
      %
      1 &  4 & 614 & 615 & 616 & 617 & 618 & 651 & 652 & 685 & 686 & 687 & 720 & 721 \\
      %
      1 &  5 & 613 & 646 & 648 & 649 & 650 & 683 & 684 & 717 & 718 & 719 & 752 & 753 & 754 & 787 \\
      %
      1 &  6 & 644 & 645 & 647 & 680 & 681 & 682 & 682 & 715 & 716 & 749 & 750 & 751 & 784 \\
      %
      1 &  7 & 712 & 713 & 714 & 747 & 748 & 781 & 782 & 783 \\
      %
      % row census:  {10, 12, 13, 12, 14, 13, 8}
      % Total count: 82
      %
      && \\
      %
      %
      2 &  1 & 658 & 690 & 787 \\
      %
      2 &  2 & 625 & 657 & 689 & 721 & 754 \\
      %
      2 &  3 & 624 & 656 & 688 & 720 & 753 \\
      %
      2 &  4 & 591 & 623 & 655 & 687 & 752 & 784 \\
      %
      2 &  5 & 590 & 622 & 654 & 686 & 719 & 751 & 783 \\
      %
      2 &  6 & 589 & 621 & 653 & 685 & 718 & 750 & 782 \\
      %
      2 &  7 & 556 & 588 & 620 & 652 & 717 & 749 & 781 \\
      %
      2 &  8 & 555 & 587 & 619 & 651 & 684 & 716 & 748 & 779 \\
      %
      2 &  9 & 522 & 554 & 586 & 618 & 683 & 715 & 747 \\
      %
      2 & 10 & 520 & 553 & 585 & 617 & 650 & 682 & 714 \\
      %
      2 & 11 & 519 & 552 & 584 & 616 & 649 & 681 & 713 & 745 \\
      %
      2 & 12 & 551 & 583 & 615 & 648 & 680 \\
      %
      2 & 13 & 582 & 614 & 646 & 647 & 679 \\
      %
      2 & 14 & 613 & 645 & 678 \\
      %
      % row census:  {3, 5, 5, 6, 7, 7, 7, 8, 7, 7, 8, 5, 5, 3}
      % Total count: 83
      %
      && \\
      %
      %
      3 &  1 & 582 & 551 & 519 \\
      %
      3 &  2 & 644 & 613 & 614 & 583 & 552 & 520 \\
      %
      3 &  3 & 645 & 646 & 615 & 584 & 553 & 522 \\
      %
      3 &  4 & 647 & 648 & 616 & 585 & 554 & 555 & 678 \\
      %
      3 &  5 & 679 & 680 & 649 & 617 & 586 & 587 & 556 \\
      %
      3 &  6 & 712 & 681 & 650 & 618 & 619 & 588 & 589 \\
      %
      3 &  7 & 713 & 682 & 683 & 651 & 620 & 621 & 590 \\
      %
      3 &  8 & 745 & 714 & 715 & 684 & 652 & 653 & 622 & 591 \\
      %
      3 &  9 & 747 & 716 & 717 & 685 & 654 & 623 & 624 \\
      %
      3 & 10 & 748 & 749 & 718 & 686 & 655 & 656 & 625 \\
      %
      3 & 11 & 779 & 781 & 750 & 719 & 687 & 688 & 657 \\
      %
      3 & 12 & 782 & 751 & 752 & 720 & 689 & 658 \\
      %
      3 & 13 & 783 & 784 & 753 & 721 & 690 \\
      %
      % row census:  {3, 6, 6, 7, 7, 7, 7, 8, 7, 7, 7, 6, 5}
      % Total count: 83
      %
      && \\
      %
    \end{tabular}
  \end{center}
\label{tab:tres rows}
\end{table} }}
 
  \begin{table}[htbp]  %  T A B L E
    \caption{Excerpted data set.}
    \begin{center}
      \begin{tabular}{r r@{.}l  r@{.}l  r@{.}l}
        %        
        $k$ & \multicolumn{2}{c}{$x_{k}$} & \multicolumn{2}{c}{$y_{k}$} & \multicolumn{2}{c}{$\phi_{k}$}\\\hline
        %
         1  & $ -15 $ & $ 879001 $ & $ -16 $ & $ 365496 $ & $ -2 $ & $ 597531 $ \\
         2  & $ -14 $ & $ 749446 $ & $ -15 $ & $ 995488 $ & $ -2 $ & $ 613017 $ \\
         3  & $ -13 $ & $ 905339 $ & $ 16 $  & $ 242941 $ & $ -2 $ & $ 557543 $ \\
         %
         \vdots \\
        %
        1\,022	 & $	13	$ & $	927362	$ & $	-16	$ & $	235010	 $ & $	-2	$ & $	780323	 $ 	\\
        1\,023	 & $	14	$ & $	741765	$ & $	15	$ & $	957687	 $ & $	-2	$ & $	687929	 $ 	\\
        1\,024	 & $	15	$ & $	905518	$ & $	16	$ & $	346979	 $ & $	-2	$ & $	599001	 $ 	\\
        %
      \end{tabular}
    \end{center}
  %\label{tab:?}
  \end{table}%

\section{Results} %    S    S    S    S    S    S    S    S    S    S    S    S

\subsection{Least Squares Results}  %   SS   SS   SS   SS   SS   SS   SS   SS   SS   SS   SS   SS

\begin{figure}[htbp] %  figure placement: here, top, bottom, or page
   \centering
   \includegraphics[ width = 4.5in ]{\pathgraphics "patterns"/"lines"/"three lines"} 
   \caption{Solutions for three data sets.}
   \label{fig:three lines}
\end{figure}

  \begin{table}[htbp]  %  T A B L E
    \caption{Least squares results for three axes.}
    \begin{center}
      \begin{tabular}{c r@{.}l c r@{.}l r@{.}l c r@{.}l r@{.}l c r@{.}l r@{.}l}
        %
        axis & \multicolumn{5}{c}{gap} & \multicolumn{5}{c}{intercept} & \multicolumn{5}{c}{slope} & \multicolumn{2}{c}{$\sqrt{\inner{r^{2}}}$} \\\hline
        %
        1 &  0 & 9899  & $\pm$ &  0 & 0032  & 3 & 438    & $\pm$ &  0 & 013  &  0 & 3376     & $\pm$ &  0 & 0017 & 0 & 052\\
        %
        2 &  4 & 974   & $\pm$ &  0 & 052   & $-$2 & 075 & $\pm$ &  0 & 093  &  5 & 168      & $\pm$ &  0 & 052 & 0 & 18 \\
        %
        3 &  1 & 2322  & $\pm$ &  0 & 0039  & $-$7 & 505 & $\pm$ &  0 & 043  &  $-$0 & 8576  & $\pm$ &  0 & 0038 & 0 & 054 \\
        %
      \end{tabular}
    \end{center}
  %\label{tab:?}
  \end{table}%

  \begin{table}[htbp]  %  T A B L E
    \caption{Intermediate results: angles for the axes.}
    \begin{center}
      \begin{tabular}{c r@{.}l c r@{.}l c rcl}
        %
        axis & \multicolumn{2}{c}{$\theta$} & $\pm$ & \multicolumn{2}{c}{$\sigma_{\theta}$} \\\hline
        %
        1 &   0 & 3256  & $\pm$ &  0 & 0015 & = & $(18.655$ & $\pm$ & $0.086)^{\circ}$\\
        %
        2 &   1 & 380   & $\pm$ &  0 & 018  & = & $(79.0$ & $\pm$ & $1.0)^{\circ}$ \\
        %
        3 & $-$0 & 7089  & $\pm$ &  0 & 0025 & = & $(-40.62$ & $\pm$ & $0.14)^{\circ}$
        %
      \end{tabular}
    \end{center}
  %\label{tab:?}
  \end{table}%

\subsection{Apex Angles}  %   SS   SS   SS   SS   SS   SS   SS   SS   SS   SS   SS   SS

  \begin{table}[htbp]  %  T A B L E
    \caption{Final results: apex angle measurements}
    \begin{center}
      \begin{tabular}{c r@{.}l c r@{.}l c r@{.}l c r@{.}l}
        %
        & \multicolumn{2}{c}{$\theta$} & $\pm$ & \multicolumn{2}{c}{$\sigma_{\theta}$} \\\hline
        %
        $\alpha$ & $1$ & $040$  & $\pm$ & $0$ & $018$  & = & $(59$ & $6$ & $\pm$ & $1$ & $0)^{\circ}$\\
        %
        $\beta$  & $1$ & $0345$ & $\pm$ & $0$ & $0029$ & = & $(59$ & $27$ & $\pm$ & $0$ & $17)^{\circ}$\\
        %
        $\gamma$ & $1$ & $041$  & $\pm$ & $0$ & $018$  & = & $(59$ & $7$ & $\pm$ & $1$ & $0)^{\circ}$\\\arrayrulecolor{medgray}\hline
        %
        total    & $3$ & $116$  & $\pm$ & $0$ & $026$  & = & $(178$ & $5$ & $\pm$ & $1$ & $7)^{\circ}$
        %
      \end{tabular}
    \end{center}
  \label{tab:apex angles}
  \end{table}%

\begin{figure}[htbp] %  figure placement: here, top, bottom, or page
   \centering
   \includegraphics[ width = 4in ]{\pathgraphics "patterns"/"lines"/"apex angles"} 
   \caption{Apex angles displayed in table \ref{tab:apex angles}.}
   \label{fig:pattern needles}
\end{figure}

\subsection{Qualitative Results}  %   SS   SS   SS   SS   SS   SS   SS   SS   SS   SS   SS   SS

\begin{figure}[htbp] %  figure placement: here, top, bottom, or page
   \centering
   \hspace{-2in}\includegraphics[ width = 2.35in ]{\pathgraphics "patterns"/"lines"/"patterns lines merit bullseye 1"} \\[5pt]
   \includegraphics[ width = 2.35in ]{\pathgraphics "patterns"/"lines"/"patterns lines merit bullseye 2"} \\[5pt]
   \hspace{2in} \includegraphics[ width = 2.35in ]{\pathgraphics "patterns"/"lines"/"patterns lines merit bullseye 3"} 
   \caption{Merit functions for the three data sets.}
   \label{fig:3 bullseyes}
\end{figure}

\endinput