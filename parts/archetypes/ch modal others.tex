\chapter{Solution Via Other Methods}

For comparison, other solution methods are applied to the Bevington data. These methods will reinforce the geometry of least squares.

\section{\label{sec:normal II}Normal Equations from Vectors}  %    S    S    S    S    S    S    S    S    S
In \S \ref{sec:normal I}, the normal equations appeared from the linear system resulting from the two partial differential equations in \eqref{eq:bev pde}. Another approach starts with the linear equation $\axeb$ and uses column vectors. The data is posed in terms of the column vectors in table \ref{tab:bevington data and results}. These, plus a constant vector, are the elements of composition:
  \begin{equation*}   %  =   =   =   =   =
    {\bf{1}} = \mat{c}{1 \\ 1 \\ 1 \\ 1 \\ 1 \\ 1 \\ 1 \\ 1 \\ 1}, \quad
    x        = \mat{c}{1 \\ 2 \\ 3 \\ 4 \\ 5 \\ 6 \\ 7 \\ 8 \\ 9}, \quad
    T        = \frac{1}{10} \mat{c}{156 \\ 175 \\ 366 \\ 438 \\ 582 \\ 616 \\ 642 \\ 704 \\ 988}.
    %\label{eq:}
  \end{equation*}
There are two column vectors in the system matrix $\A{}$:
  \begin{equation*}   %  =   =   =   =   =
    \A{} = 
      \mat{c|c}{{\bf{1}} & x}
    %\label{eq:}
  \end{equation*}
In the service of a crisp mental image, the linear system is written explicitly:
  \begin{equation}   %  =   =   =   =   =
  \begin{array}{cccc}
      \A{} & a & = & T \\
      \mat{cc}{
         1 & 1 \\
         1 & 2 \\
         1 & 3 \\
         1 & 4 \\
         1 & 5 \\
         1 & 6 \\
         1 & 7 \\
         1 & 8 \\
         1 & 9 } &
      \mat{c}{ a_{0} \\ a_{1} } & = &
         \frac{1}{10} \mat{c}{156 \\ 175 \\ 366 \\ 438 \\ 582 \\ 616 \\ 642 \\ 704 \\ 988} .
    \label{eq:bevington axeb}
  \end{array}
  \end{equation}

\subsection{Composing the normal equations}  %   SS   SS   SS   SS   SS   SS   SS   SS   SS   SS   SS   SS
The system encodes nine linear problems:
  \begin{equation*}   %  =   =   =   =   =
   \begin{split}
    a_{0} + a_{1} x_{1} & = T_{1}, \\
     & \ \, \vdots \\
    a_{0} + a_{1} x_{9} & = T_{9}.
    %\label{eq:}
    \end{split}
  \end{equation*}
But there are no parameters $a_{0}$ and $a_{1}$ which solve all nine equations. More formally, the data vector $T$ is not in the column space of $\A{}$. There are no parameters $a_{0}$ and $a_{1}$ such that
  \begin{equation*}   %  =   =   =   =   =
    a_{0} {\bf{1}} + a_{1} x = T.
  \end{equation*}
Therefore there is no exact solution $a$ such that
  \begin{equation*}   %  =   =   =   =   =
   %\begin{split}
      \normts{\A{}a - T} = 0 .
   %\end{split}
   %\label{eq:}
  \end{equation*}
The root cause is that the data vector has components in the range space and in the null space:
  \begin{equation*}   %  =   =   =   =   =
   %\begin{split}
      T = {\bl{T_{\atomrng}}} + {\rd{T_{\atomnll}}}.
   %\end{split}
   %\label{eq:}
  \end{equation*}


One strategy is to compose a problem does have a solution
  \begin{equation*}   %  =   =   =   =   =
    \wx{*} a = \A{*} T.
  \end{equation*}
Certainly the vector $\A{*} T$ is in the column space of $\A{*}$ -- the coordinates are $T$! The solution is
  \begin{equation*}   %  =   =   =   =   =
    \mat{c}{a_{0} \\ a_{1}} = \wxi{*} \A{*}T.
  \end{equation*}
For the Bevington data set the dot products are,
  \begin{equation*}   %  =   =   =   =   =
  \begin{split}
    \oto &= m = 9, \\
    \otx &= \xto = 45, \\
    \xtx &= 285, \\
    {\bf{1}}^{\mathrm{T}}T &= \frac{4667}{10}, \\
    x^{\mathrm{T}}T        &= 2898.
    %\label{eq:}
  \end{split}
  \end{equation*}
Other intermediate products include the product matrix,
  \begin{equation*}   %  =   =   =   =   =  
     \wx{*} = \mat{cc}{\oto & \otx \\ \xto & \xtx} = \mat{cc}{9 & 45 \\ 45 & 285},
  \end{equation*}
its determinant,
  \begin{equation}   %  =   =   =   =   =
    \Delta = \paren{\oto} \paren{\xtx} - \paren{\otx}^{2},
    \label{eq:det again}
  %\end{split}
  \end{equation}
inverse,
  \begin{equation*}   %  =   =   =   =   =
  %\begin{split}
    \wxi{*} = \Delta^{-1} 
      \mat{rr}{\xtx & -\oto \\ -\oto & \otx}
    %\label{eq:}
  %\end{split}
  \end{equation*}
and the vector,
  \begin{equation*}   %  =   =   =   =   =
    \A{*}T = 
      \mat{c}{{ \bf{1}}^{\mathrm{T}}T \\ x^{\mathrm{T}}T } =
      \frac{1}{10} \mat{c}{4667 \\28\,980}.
  \end{equation*}
The solution is again
  \begin{equation*}   %  =   =   =   =   =
  %\begin{split}
    a = \Delta^{-1} 
      \mat{rr}{\xtx & -\oto \\ -\oto & \otx} 
      \mat{c}{{ \bf{1}}^{\mathrm{T}}T \\ x^{\mathrm{T}}T }
    %\label{eq:}
  %\end{split}
  \end{equation*}
  \begin{equation}   %  =   =   =   =   =
  %\begin{split}
    \mat{c} {a_{0} \\ a_{1}} = \Delta^{-1}
    \mat{c}{ \paren{\xtx} \paren{\mathbf{1}^{\mathrm{T}}T} -
             \paren{\otx} \paren{x^{\mathrm{T}}T} \\
             \paren{\oto} \paren{x^{\mathrm{T}}T} -
             \paren{\otx} \paren{\mathbf{1}^{\mathrm{T}}T} }
    \label{eq:bevington:soln:vectors}
  %\end{split}
  \end{equation}
leading to the same solution presented in \S \ref{sec:exact form}.

\section{$\Q{}\R{}$ Decomposition}  %    S    S    S    S    S    S    S    S    S
Resolve the range space $\brnga{}$ into an orthonormal basis.
  \begin{equation*}   %  =   =   =   =   =
    \A{} = \Q{} \R{}
  \end{equation*}
  \begin{equation*}   %  =   =   =   =   =
    a = \R{-1}\Q{*}T
  \end{equation*}
  
\subsection{Computing the QR Decomposition}  %   SS   SS   SS   SS   SS   SS   SS   SS   SS   SS   SS   SS
$\R{}\in\real{\byy{\rho}}$
  \begin{equation*}   %  =   =   =   =   =
    \R{} = \mat{cc}{ r_{11} & r_{12} \\ 0 & r_{22} } = \mat{cc}{\nu_{1} & q_{1}^{2}a_{2} \\ 0 & \nu_{2}}
  \end{equation*}
The norm of the first column vector:
  \begin{equation}
        r_{11} = \normt{a_{1}} = 3
    %\end{split}
    %\label{eqn:}
  \end{equation}
Normalized column vector
  \begin{equation}
        q_{1} = \frac{a_{1}} {r_{11}} = \frac{1}{3}\mat{c}{1\\1\\1\\1\\1\\1\\1\\1\\1}
    %\end{split}
    %\label{eqn:}
  \end{equation}
Next scale factor in row 1:
  \begin{equation}
        r_{12} = q_{1}^{*}a_{2} = 15
    %\end{split}
    %\label{eqn:}
  \end{equation}
% = =
% = =  e q u a t i o n
  \begin{equation}
        q_{2} = a_{2} - r_{12} q_{1}
    %\end{split}
    %\label{eqn:}
  \end{equation}
% = =
  \begin{equation*}   %  =   =   =   =   =
    \Q{} = 
       \mat{cc}{ \frac{1}{3} 
       \mat{c}{1 \\ 1 \\ 1 \\ 1 \\ 1 \\ 1 \\ 1 \\ 1 \\ 1} &  \frac{1}{2\sqrt{15}}
       \mat{r}{ -4 \\ -3 \\ -2 \\ -1 \\ 0 \\ 1 \\ 2 \\ 3 \\ 4 } }
  \end{equation*}
  \begin{equation*}   %  =   =   =   =   =
    \begin{array}{cccc}
      \A{} & = & \Q{} & \R{}, \\
      \mat{cc}{
         1 & 1 \\
         1 & 2 \\
         1 & 3 \\
         1 & 4 \\
         1 & 5 \\
         1 & 6 \\
         1 & 7 \\
         1 & 8 \\
         1 & 9 } & = &
       \mat{cc}{ \frac{1}{3} 
       \mat{c}{1 \\ 1 \\ 1 \\ 1 \\ 1 \\ 1 \\ 1 \\ 1 \\ 1} &  \frac{1}{2\sqrt{15}}
       \mat{r}{ -4 \\ -3 \\ -2 \\ -1 \\ 0 \\ 1 \\ 2 \\ 3 \\ 4 } } &
       \mat{cc}{ 3 & 15 \\ 0 & 2\sqrt{15} }         
    %\label{eq:}
    \end{array}
  \end{equation*}

\endinput