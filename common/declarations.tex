%    As set up here, all theorem-class objects will be numbered with
%    the same counter, starting with 1 at every new chapter; numbers
%    will have the form <chapter>.<theorem>.  This may be changed if
%    the author prefers.
%\newtheorem{myDefinition}{Definition}
%\newtheorem{myTheorem}{Theorem}
%\newtheorem{myLemma}{Lemma}
%\newtheorem{theorem}{Theorem}[chapter]
%\newtheorem{lemma}[theorem]{Lemma}
%
%\theoremstyle{definition}
%\newtheorem{definition}[theorem]{Definition}
%\newtheorem{example}[theorem]{Example}
%\newtheorem{xca}[theorem]{Exercise}
%\newtheorem{myTheorem}{Theorem}

%\theoremstyle{remark}
%\newtheorem{remark}[theorem]{Remark}

% hierarchal numbering
\numberwithin{section}{chapter}
\numberwithin{equation}{chapter}
\numberwithin{table}{chapter}
\numberwithin{figure}{chapter}

% other definitions
\input{\fullpath A}
\input{\fullpath abbreviations}
\input{\fullpath colors}
\input{\fullpath delimiters}
\input{\fullpath derivatives}
\input{\fullpath "example a matrices"}
\input{\fullpath "example b matrices"}
\input{\fullpath "example c matrices"}
\input{\fullpath exemplars}
\input{\pathname "bold letters"}
\input{\fullpath ftola}
\input{\pathname "least squares"}
\input{\fullpath norms}
\input{\pathname spaces}
\input{\pathname "matrix basics"}
%\input{\pathname "matrix decompositions"}
\input{\pathname "sigma matrices"}
\input{\pathname "svd forms"}
\input{\pathname "tikz constants.tex"}
\input{\pathname vectors}

% leftovers
\newcommand{\recip}[1]   {\frac{1}{#1}}
\newcommand{\half}[0]    {\tfrac{1}{2}}
\newcommand{\sq}[1]      {\paren{#1}^{2}}
\newcommand{\org}[0]     {O}
\newcommand{\merit}[1]   {\chi^{2}\paren{#1}}
\newcommand{\rtr}[1]     {r^{\mathrm{#1}}r}
\newcommand{\kto}[1]     { k = 1\! :\! {#1}}
\newcommand{\ktoo}[2]    {#1 = 1\! :\! {#2}}

% Hilbert spaces
\newcommand{\ltwo}       {\mbox{$l^{2}$}}
\newcommand{\Ltwo}       {\mbox{$L^{2}$}}

% stray commands
\newcommand{\oto}[0]     { \mathbf{1}^{\mathrm{T}} \mathbf{1} }
\newcommand{\xto}[0]     {          x^{\mathrm{T}} \mathbf{1} }
\newcommand{\otx}[0]     { \mathbf{1}^{\mathrm{T}} x }
\newcommand{\xtx}[0]     {          x^{\mathrm{T}} x }
\newcommand{\oty}[0]     { \mathbf{1}^{\mathrm{T}} y }
\newcommand{\xty}[0]     {          x^{\mathrm{T}} y }
\newcommand{\ott}[0]     { \mathbf{1}^{\mathrm{T}} T }
\newcommand{\xtt}[0]     {          x^{\mathrm{T}} T }

\newcommand{\poto}[0]    { \paren{\mathbf{1}^{\mathrm{T}} \mathbf{1}} }
\newcommand{\pxto}[0]    {          \paren{x^{\mathrm{T}} \mathbf{1}} }
\newcommand{\potx}[0]    { \paren{\mathbf{1}^{\mathrm{T}} x} }
\newcommand{\pxtx}[0]    {          \paren{x^{\mathrm{T}} x} }
\newcommand{\poty}[0]    { \paren{\mathbf{1}^{\mathrm{T}} y} }
\newcommand{\pxty}[0]    {          \paren{x^{\mathrm{T}} y} }
\newcommand{\pott}[0]    { \paren{\mathbf{1}^{\mathrm{T}} T} }
\newcommand{\pxtt}[0]    {          \paren{x^{\mathrm{T}} T} }

\newcommand{\mydet}[0]   { \poto \pxtx - \potx^{2} }
\newcommand{\myslope}[0] { \Delta^{-1} \paren{\poto \pxtt - \potx \pott} }
\newcommand{\myint}[0]   { \Delta^{-1} \paren{\pxtx \poty - \potx \pxty} }

\newcommand{\cost}[0]    { \cos \theta }
\newcommand{\sint}[0]    { \sin \theta }
\newcommand{\costs}[0]   { \cos^{2} \theta }
\newcommand{\sints}[0]   { \sin^{2} \theta }
\newcommand{\cst}[0]     { \cos \theta \sin \theta }
\newcommand{\thatmat}[0] { \mat{cc}{ \oto & \otx \\ \xto & \xtx } }

%\newcommand{\udenp}[2]   { \sqrt{#1 + \frac{#2} {\sqrt{2341}}} }
%\newcommand{\udenm}[2]   { \sqrt{#1 - \frac{#2} {\sqrt{2341}}} }
\newcommand{\udenp}[2]   { \sqrt{#1 + #2 / \sqrt{2341}} }
\newcommand{\udenm}[2]   { \sqrt{#1 - #2 / \sqrt{2341}} }
\newcommand{\urowmmp}[2] { \udenm{#1}{#2} & -\udenp{#1}{#2} }
\newcommand{\urowpmm}[2] { \udenp{#1}{#2} & -\udenm{#1}{#2} }
\newcommand{\urowppm}[2] { \udenp{#1}{#2} &  \udenm{#1}{#2} }

% bevington solution
\newcommand{\bsoln}[0]   {\frac{1}{360} \mat{c}{1733\\3387}}
\newcommand{\bsolna}[0]  {\mat{c}{4.81389\\9.40833}}

% bound angle
\newcommand{\bangle}[1]  { 0 \le #1 < 2\pi }

% order
\newcommand{\order}[1]   { \mathcal{o}\paren{#1} }
\newcommand{\Order}[1]   { \mathcal{O}\paren{#1} }

\newcommand{\mx}[2]      { \max\limits_{#1 \in \cmplx{ #2 }} }
\newcommand{\mn}[2]      { \min\limits_{#1 \in \cmplx{ #2 }} }

\newcommand{\mxxn}[0]    { \mx{x}{n} }
\newcommand{\mnxn}[0]    { \mn{x}{n} }

\newcommand{\mxball}[0]  { \max\limits_{\normt{x} = 1} }

% is equal?
\newcommand{\iseq}[0]    { \overset{?}{=} }

% map A
\newcommand{\mapa}[1]    { \overset{\A{#1}}{\mapsto} }

% header
\newcommand{\head}[1]    { $\A{}$ & $=$ & $\U{}$ & $\sig{}$ & $\V{*}$ }

% three lines
\newcommand{\factor}[0]  {\sqrt{m^{4} - 8m + 8}}

% spacings and such
\newcommand{\wfour}[0]   {1.05in}

\endinput  %  -  -  -  -  -  -  -  -  -  -  -  -  -  -  -  -  -  -  -  -