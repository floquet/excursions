\documentclass[10pt]{newsiambook}

\usepackage{amsmath,amssymb}
%\usepackage{epstopdf} 
\usepackage{color}
\usepackage{xcolor}
%\usepackage{transparent}
\usepackage{colortbl}
\setlength\minrowclearance{4pt}
%\usepackage{graphicx}
%\usepackage{pifont}
\usepackage{makeidx}
%\usepackage{multicol}
\usepackage{multirow}
\usepackage{crop}
\usepackage{pdfpages}
\usepackage{lscape}
%\usepackage{pdflscape}
%\usepackage{pdflscape}
%\usepackage{tabularx}
%\usepackage{pstricks}
%\usepackage{sidecap}
\usepackage{stmaryrd}  % https://tex.stackexchange.com/questions/26508/left-version-of-mapsto
%\usepackage{epstopdf}
\usepackage{siunitx}
%https://tex.stackexchange.com/questions/7735/how-to-get-straight-quotation-marks
%\usepackage[T1]{fontenc}
%\usepackage{upquote}
\usepackage{textcomp}  % http://hevea.inria.fr/examples/test/sym.html
\usepackage{tikz}
\usepackage{overpic}
\usepackage[nomessages]{fp} % http://tex.stackexchange.com/questions/30081/how-can-i-sum-two-values-and-store-the-result-in-other-variable
 
\crop	
\newcommand{\pathname}{../"common"/}
\newcommand{\fullpath}{\pathname}
\newcommand{\pathgraphics}{../pdf/}

\input{\pathname declarations.tex}
	
\makeindex

\begin{document}

\frontmatter
%
\tableofcontents
%
\listoffigures
%%
\listoftables
%
%\input{chapters/frontmatter/preface}
%\input{chapters/frontmatter/literature}

\input{../frontmatter/"ch literature"}

\mainmatter
%%%%%%%
\part{Rudiments}
\input{../chapters/rudiments/"ch least squares problems"}
\input{../chapters/rudiments/"ch least squares solutions"}
%\input{../chapters/rudiments/"ch grid"}

%%%%%%%
\part{Modal Example}
Solving for the modes in a expansion basis. Here linear regression.
\input{../chapters/archetypes/"ch modal calculus"}
\input{../chapters/archetypes/"ch modal svd"}
\input{../chapters/archetypes/"ch modal others"}
\input{../chapters/archetypes/"ch modal finer points"}
\input{../chapters/archetypes/"ch modal tricks"}

%%%%%%%
\part{Zonal Example}
\input{../chapters/archetypes/"ch archetype zonal"}

%%%%%%%
%\part{Least Squares Theory}

%%%%%%%
%\part{Applications: Polynomial Approximation}

%%%%%%%
%\part{Applications:\\Finding Patterns}
%\input{../chapters/patterns/"ch patterns lines"}
%\input{../chapters/patterns/"ch patterns crystals"}

%%%%%%%
%\part{Applications: Stitching}
%\input{../chapters/stitch/"ch stitch piston"}

%%%%%%%
%\part{Applications:\\Inverting the Gradient}
%\input{../chapters/gradient/"ch gradient"}

%%%%%%%
\part{Applications:\\Nonlinear Problems}
\input{../chapters/nonlinear/"ch linearization"}
\input{../chapters/nonlinear/"ch census"}


%%%%%%%%
\part{Appendices}
\appendix
\input{../appendices/"app exercises"}
%\input{../appendices/"app exemplars"}
%\input{../appendices/"app errors"}
%\input{../appendices/"app notation"}
%\input{../appendices/"app jargon"}

%%%%%%
\part{\ \ Backmatter}
\backmatter
%\input{backmatter/acronyms}
%\input{chapters/backmatter/notation/"head notation"}
%%\input{backmatter/glossary}
%\bibliography{chapters/backmatter/bibliography/svdfp}
%\input{chapters/backmatter/bibliography/svdfp.bib}

\input{../"ells bibliography"}

\printindex

\end{document}  %   .   .   .   .   .   .   .   .   .   .   .   .   .   .   .   .   .   .

%  \tiny
%  \scriptsize
%  \footnotesize
%  \small
%  \normalsize
%  \large
%  \Large
%  \LARGE
%  \huge
%  \Huge

% \section{sec} %    S    S    S    S    S    S    S    S    S    S    S    S
% \subsection{ssec}  %   SS   SS   SS   SS   SS   SS   SS   SS   SS   SS   SS   SS
